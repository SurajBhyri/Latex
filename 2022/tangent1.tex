\begin{enumerate}

	\item Draw a circle of radius 2.5 cm. Take a point P outside the circle at a distance of 7 cm from the center. Then construct a pair of tangents to the circle from point P.

	\item Write the steps of construction for constructing a pair of tangents to a circle of radius 4 cm from a point P, at a distance of 7 cm from its center O.

	\item In Figure \ref{fig:tan1}, there are two concentric circles with centre O. If ARC and AQB are tangents to the smaller circle  from the point A lying on the larger circle, find the length of AC, if AQ = 5 cm.
		\begin{figure}[H]
			\centering
			\includegraphics[width=\columnwidth]{tan}
			\caption{Two concentric circles with O as centre}
			\label{fig:tan1}
		\end{figure}
	
	\item In Figure \ref{fig:cir1}, if a circle touches the side QR of $\Delta$PQR at S and extended sides PQ and PR at M and N, respectively,
		\begin{figure}[H]
			\centering
			\includegraphics[width=\columnwidth]{cir}
				\caption{Circle touching a triangle with two extended sides as tangents to the circles}
				\label{fig:cir1}
		\end{figure}
		prove that $PM=\dfrac{1}{2}(PQ+QR+PR)$
	\item In Figure \ref{fig:tri1}, a triangle ABC is drawn to circumscribe a circle of radius 4 cm such that the segments BD and DC into which BC is divided by the point of contact D are of lengths 6 cm and 8 cm respectively. If the area of $\Delta$ABC is 84 $cm^2$, find the lengths of sides AB and AC.
		\begin{figure}[H]
			\centering
			\includegraphics[width=\columnwidth]{tri}
				\caption{Circle with O as center circumscribed in triangle ABC}
				\label{fig:tri1}
		\end{figure}
	\item In Figure \ref{fig:sq1}, PQ and PR are tangents to the circle centered at O. If $\angle OPR=45\degree$, then prove that ORPQ is a square.
		\begin{figure}[H]
			\centering
			\includegraphics[width=\columnwidth]{sq}
			\caption{Two tangents drawn from point P to a circle whose centre is O}
			\label{fig:sq1}
		\end{figure}
	\item In Figure \ref{fig:sct1}, O is the centre of a circle of radius 5 cm. PA and BC are tangents to the circle at A and B respectively. If OP is 13 cm, then find the length of tangents PA and BC.
		\begin{figure}[H]
			\centering
			\includegraphics[width=\columnwidth]{sct}
			\caption{Two tangents drawn from point C to a circle whose centre is O}
			\label{fig:sct1}
		\end{figure}

	\item In Figure \ref{fig:ver1}, AB is diameter of a circle centered at O. BC is tangent to the circle at B. If OP bisects the chord AD and $\angle AOP=60\degree$, then find $m\angle C$.
		\begin{figure}[H]
			\centering
			\includegraphics[width=\columnwidth]{ver}
			\caption{Tangent BC is drawn from point C to a circle whose centre is O}
			\label{fig:ver1}
		\end{figure}

	\item In Figure \ref{fig:hor1}, XAY is a tangent to the circle centered at O. If $\angle ABO=60\degree$,then find $m\angle BAY$ and $m\angle AOB$.
		\begin{figure}
			\centering
			\includegraphics[width=\columnwidth]{hor}
			\caption{The line XAY is tangent to the circle centered at O}
			\label{fig:hor1}
		\end{figure}

	\item Two concentric circles are of radii 4cm and 3 cm. Find the length of the chord of the larger circle which touches the smaller circle.

	\item In Figure \ref{fig:sl1}, a triangle ABC with $\angle B=90\degree$ is shown. Taking AB as diameter, a circle has been drawn intersecting AC at point P. Prove that the tangent drawn at point P bisects BC.
		\begin{figure}[H]
			\centering
			\includegraphics[width=\columnwidth]{sl}
			\caption{PQ is tangent to the circle centered at O. AB is the diameter and $\angle B=90\degree$}
			\label{fig:sl1}
		\end{figure}
\end{enumerate}
